\section{Introducci\'on}
El modelo cl\'asico de redes aleatorias de Erd\"os-R\`enyi asume que un par de 
nodos est\'an conectados con una probabilidad $p$, generando redes estadisticamente
homog\'eneas y distribuci\'on de grado tipo Poisson~\citep{jeong2000}. Sin embargo, 
muchas redes complejas reales (como internet~\citep{faloutsos1999} o 
redes metab\'olicas~\citep{jeong2000}) muestran comportamientos libre de 
escala, es decir, se caracterizan por tener pocos nodos altamente conectados 
\textit{hubs} con nodos poco conectados.

En el caso de redes biol\'ogicas, a trav\'es de t\'ecnicas gen\'eticas, se ha 
demostrado la existencia de genes indispensables para la sobrevivencia (genes
\textit{esenciales}) \citep{kamath2003,winzeler1999}. Desde entonces se ha 
buscado una forma de caracterizar la esencialidad de un nodo en una red a 
trav\'es de sus caracter\'isticas topol\'ogicas.

En \citet{jeong2000} se trabaja sobre 43 redes metab\'olicas y reporta una 
correlaci\'on entre los hubs de la red y la esencialidad del nodo desde 
el punto de vista biol\'ogico. A esta correlaci\'on la llamaron 
\textit{regla de Centralidad-Letalidad} (Centrality-Lethality rule) y 
posteriormente este trabajo ha sido reportado numerosas veces por otros 
autores. La principal cr\'itica que ha recibido esta correlaci\'on es al ser
considerada muchas veces como raz\'on causal, es por ello que en
\citet{he2006} se plantea una aproximaci\'on distinta al problema, pero 
con consecuencias equivalentes. En \citet{he2006} se establece que la
esencialidad de un nodo (i.e. prote\'ina) se debe a la participaci\'on de 
este en una interacci\'on/proceso (conexiones) esencial, luego debido 
a que los hubs tienen una alta conectividad con otros nodos, es m\'as 
probable que una de sus interacciones sea esencial.


El presente estudio es una revisi\'on del trabajo de \citet{zotenko2008} en que 
se estudia las debilidades de las hip\'otesis de Jeong y He. Para ello utilizamos
cuatro redes proteicas de levadura (\textit{S. cerevisiae}): Obtenida a partir 
de Affinity-Purification/Mass-Spectrometry (AP-MS) \citep{apms_data}, de 
interacciones binarias yeast two-hybrid (Y2H) \citep{y2h_data} y dos curadas 
de literatura \citep{lit_data} {\bf (FALTA UNA CITA AQUI)}. Los datos para 
construir las redes fueron obtenidos de la \textit{Yeast Interactome Database}.

\section{Introducci\'on}
El modelo clasico de redes aleatorias de Erd\"os-R\`enyi asume que un par de 
nodos est\'an conectados con una probabilidad $p$, generando redes estadisticamente
homogenias y distribuci\'on de grado tipo Poisson~\citep{jeong2000}. Sin embargo 
muchas redes complejas reales (como internet~\citep{faloutsos1999} o 
redes metabolicas~\citep{jeong2000}) muestran comportamientos libre de 
escala, es decir, se caracterizan por tener pocos nodos altamente conectados 
\textit{hubs} con nodos poco conectados.

En el caso de redes biologicas, a trav\'es de tecnicas gen\'eticas, se ha 
demostrado la existencia de genes indispensables para la sobrevivencia (genes
\textit{esenciales}) \citep{kamath2003,winzeler1999}. Desde entonces se ha 
buscado una forma de caracterizar la escencialidad de un nodo en una red a 
trav\'es de sus caracteristicas topologicas.

En \citet{jeong2000} se trabaja sobre 43 redes metabolicas y reporta una 
correlaci\'on entre los hubs de la red y la escencialidad del nodo desde 
el punto de vista biol\'ogico. A esta correlaci\'on la llamaron 
\textit{regla de Centralidad-Letalidad} (Centrality-Lethality rule) y 
posteriormente a este trabajo ha sido reportada numerosas veces por otros 
autores. La principal critica que ha resivido esta correlaci\'on es al ser
considerada muchas veces como raz\'on causal, es por ello que en
\citet{he2006} se planetea una aproximaci\'on distinta al problema, pero 
con consecuencias equivalentes. En \citep{he2006} se establece que la
escencialidad de un nodo (i.e. prote\'ina) se debe a la participaci\'on de 
este en una interacci\'on/proceso (conexiones) escencial, luego debido 
a que los hubs tienen una alta conectividad, con otros nodos, es m\'as 
probable una de sus interacciones sea escencial.


El presente trabajo es una revisi\'on del trabajo de \citet{zotenko2008} en que 
se estudia las debilidades de las hipotesis de Jeong y He. Para ello utilizamos
cuatro redes proteicas de levadura (\textit{S. cerevisiae}): Obtenida a partir 
de Affinity-Purification/Mass-Spectrometry (AP-MS) \citep{apms_data}, de 
interacciones binarias yeast two-hybrid (Y2H) \citep{y2h_data} y dos curadas 
de literatura \citep{lit_data} {\bf (FALTA UNA CITA AQUI)}. Los datos para 
contruir las redes fueron obtenidos de la \textit{Yeast Interactome Database}\footnote{http://interactome.dfci.harvard.edu/S\_cerevisiae/}
